\documentclass[../main.tex]{subfiles}
\begin{document}
\subsection{Experimento 1}
Primero obtenemos el volumen del $H_2$.
Usando:
\begin{equation} \label{pressure_eq}
    P_{atm} = P_{H_2} + P_{Columna \, de \, H_2O}
\end{equation}
Reemplazando en \ref{pressure_eq}:
\[ 10.33mH_2O = P_{H_2} + 0.079mH_2O \]
\[ P_{H_2} = 10.251mH_2O \]
Ahora usamos la Ley general de los gases:
\begin{equation} \label{gas_eq}
    \frac{P_1 \cdot V_1}{T_1} = \frac{P_{CN} \cdot V_{CN}}{T_{CN}}
\end{equation}
Reemplazando en \ref{gas_eq0:}
\[
    \frac{10.251 \, mH_2O \cdot 24 \, mL}{295 \, K} =
    \frac{10.33 \, mH_2O \cdot V_{CN}}{273 \, K}
\]
\[ V_{CN} = 22.04 \, mL\]
En segundo lugar, debemos hallar el #moles $(\eta)$ del $H_2$.
Usando la reacción:
\[ Mg \, + \, HCl \rightarrow \, H_2 \, + \, MgCl_2 \]

\subsection{Experimento 2}

Usando la ley de Graham \ref{diffusion_eq}:
\begin{equation}
    \frac{v_{gas\, 1}}{v_{gas\, 2}} =
    \sqrt{\frac{M_{gas\, 2}}{M_{gas\, 1}}} =
    coeficiente \, de \, difusi\acute{o}n \, de \, Graham
\end{equation}
\[
    \frac{\frac{l_{gas\, 1}}{\Delta t_1}}{\frac{l_{gas\, 2}}{\Delta t_2}} =
    \sqrt{\frac{M_{gas\, 2}}{M_{gas\, 1}}} 
\]
Pero dado que los tiempos $t_1$ y $t_2$ son iguales:
\[
    \frac{l_{gas\, 1}}{l_{gas\, 2}} =
    \sqrt{\frac{M_{gas\, 2}}{M_{gas\, 1}}} =
    \Lambda
\]
Sabiendo que:
\begin{itemize}
    \item $\overline{M}_{NH_3}$ = 17 $g/mol$
    \item $\overline{M}_{HCl}$  = 36.5 $g/mol$
\end{itemize}
Reemplazando las masas molares para obtener $\Lambda_{te\acute{o}rico}$:
\[
    \sqrt{\frac{M_{HCl}}{M_{NH_3}}} =
    \Lambda_{te\acute{o}rico}
\]
\[
    \sqrt{\frac{36.5}{17}} =
    \Lambda_{te\acute{o}rico}
\]
\[ \Lambda_{te\acute{o}rico} \approx 1.465 \]
Mientras que si realizamos los mismo para las longitudes recorridad, así 
obteniendo el $\Lambda_{real}$:
\[
    \frac{18.48 \, cm}{12 \, cm} =
    \Lambda_{real}
\]
\[ \Lambda_{real} = 1.54\]
Ahora calculamos el porcentaje de aproximación:
\[ \% \; aproximaci\acute{o}n = \frac{1.54}{1.465} \cdot 100\%\]    
\[ \% \; aproximaci\acute{o}n = 105.1\%\]
Ajustando:
\[ \% \; aproximaci\acute{o}n_{exceso} = 94.9\%\]

\end{document}