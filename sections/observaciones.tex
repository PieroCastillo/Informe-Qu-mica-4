\documentclass[../main.tex]{subfiles}

\usepackage{multirow}
\usepackage{float}
\usepackage{array}
\usepackage{tabularx}

\usepackage{tikz}
\usetikzlibrary{positioning, shapes, shadows}
\usepackage{enumitem}
\usepackage{caption}
\captionsetup[table]{name = tabla}

\begin{document}
\begin{table}[H]
    \centering
    \begin{tabular}{|l|r|}
        \hline
        Temperatura & 22°C \\
        \hline
        Presion atmosferica & 10.33 mm $H_2O$ \\
        \hline
        Volumen del gas recogido & 24 ml \\
        \hline
        Altura de la columna de agua & 7.9 cm \\
        \hline
        Longitud de la cinta de Mg & 3.3 cm \\
        \hline
        Volumen de HCL & 60ml \\
        \hline
        Concentracion molar de HCL & 6 M \\
        \hline
    \end{tabular}
    \caption{Experimento del volumen molar del hidrógeno}
    \label{tab:my_label}
\end{table}

\begin{table}[H]
    \centering
    \begin{tabular}{|l|r|}
        \hline
        Longitud precisa del tubo de Graham & 30.48 cm \\
        \hline
        Distancia recorrida por el NH3 & 18.48cm \\
        \hline
        Distancia recorrida por el HCL & 12cm \\
        \hline
    \end{tabular}
    \caption{Experimento de la Ley de difusión de Graham}
    \label{tab:my_label}
\end{table}

\end{document}