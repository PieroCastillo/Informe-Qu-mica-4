\documentclass[../main.tex]{subfiles}
\begin{document}
\subsection{Experimento 1}

\subsection{Experimento 2}

Usando la ley de Graham \ref{diffusion_eq}:
\begin{equation}
    \frac{v_{gas\, 1}}{v_{gas\, 2}} =
    \sqrt{\frac{M_{gas\, 2}}{M_{gas\, 1}}} 
\end{equation}
\[
    \frac{\frac{l_{gas\, 1}}{\Delta t_1}}{\frac{l_{gas\, 2}}{\Delta t_2}} =
    \sqrt{\frac{M_{gas\, 2}}{M_{gas\, 1}}} 
\]
Pero dado que los tiempos $t_1$ y $t_2$ son iguales:
\[
    \frac{l_{gas\, 1}}{l_{gas\, 2}} =
    \sqrt{\frac{M_{gas\, 2}}{M_{gas\, 1}}} 
\]
Sabiendo que:
\begin{itemize}
    \item $\overline{M}_{NH_3}$ = 17 $g/mol$
    \item $\overline{M}_{HCl}$  = 36.5 $g/mol$
\end{itemize}
Reemplazando para obtener $l_{HCl}$:
\[
    \frac{18.48 cm}{l_{HCl}} =
    \sqrt{\frac{36.5}{17}} 
\]
\[ l_{HCl} \approx 12.6 \, cm\]
Mientras que si realizamos los mismo para $l_{NH_3}$:
\[
    \frac{l_{NH_3}}{12 \, cm} =
    \sqrt{\frac{36.5}{17}} 
\]
\[ l_{HCl} \approx 17.6 \, cm\]
Ahora calculamos los porcentajes de aproximación
para ambos casos:
\begin{itemize}
    \item Para el $l_{HCl}$:
    \[ \% \; aproximaci\acute{o}n = \frac{12 \, cm}{12.6 \, cm} \cdot 100\%\]
    \[ \% \; aproximaci\acute{o}n = 95.23\%\]
    \item Para el $l_{NH_3}$:
    \[ \% \; aproximaci\acute{o}n = \frac{18.48 \, cm}{17.6 \, cm} \cdot 100\%\]
    \[ \% \; aproximaci\acute{o}n = 105\%\]
    Ajustando:
    \[ \% \; aproximaci\acute{o}n_{exceso} = 95\%\]
\end{itemize}

\end{document}