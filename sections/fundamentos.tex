\documentclass[../main.tex]{subfiles}
\begin{document}
\subsection{Volumen Molar $(V_m)$}
El volumen molar estándar de un gas es el volumen que ocupa una mol de gas a 
condiciones estándar. \\
Estas condiciones son la presión de 1 $atm$
y la temperatura de 273.15 $K$. \cite{lab}

\subsection{Presión $(P)$}
La presión es una magnitud física que mide la proyección de la fuerza en
dirección perpendicular por unidad de superficie, y sirve para caracterizar 
cómo se aplica una determinada fuerza resultante sobre una
línea. \cite{pression}

\subsection{Presión Atmosférica}
Se genera gracias a la gravedad que opera y hace que la atmósfera en su totalidad 
ejerza una presión sobre la superficie, su valor es de $1 \cdot 10^5$, entonces podemos 
definir a la presión atmosférica como la presión que ejerce la atmósfera de la Tierra, 
sin embargo, para ser más precisos podemos decir que el valor real de la presión 
atmosférica depende de la localización, la temperatura y las condiciones climáticas.

\subsection{Presión de un Gas}
A partir de la definición anterior podremos hablar de la presión de un gas, 
este generan una presión en cualquier superficie con la que mantengan 
contacto, esto es debido a que las moléculas gaseosas están en constante 
movimiento, además en sus colisiones, las moléculas del gas ejercen una fuerza 
sobre las paredes del recipiente.
La ley universal de los gases es de mucha utilidad para hacer cálculos con respecto 
a las propiedades de los gases como la cantidad de gas (en moles) y su volumen, 
temperatura y presión, en estos cálculos debemos trabajar con conceptos de gas 
húmedo y gas seco.\cite{chang}

\subsection{Difusión Gaseosa}
Una demostración directa del movimiento aleatorio de los 
gases la proporciona la difusión, la mezcla gradual de moléculas
de un gas con moléculas de otro debido a sus propiedades 
cinéticas. \\
A pesar de que las velocidades moleculares son muy altas, el 
proceso de difusión lleva un tiempo relativamente largo para 
completarse. Por lo tanto, la difusión de los gases 
siempre ocurre de manera gradual y no instantánea, como parecen
sugerir las velocidades moleculares. Además, debido a que 
la velocidad cuadrática media de un gas ligero es mayor que 
la de un gas más pesado,
un gas más ligero se difundirá a través de un 
espacio determinado más rápidamente que un gas más pesado. \cite{chang}

\subsection{Ley de Difusión de Graham}
Es la relación matemática presente en la difusión gaseosa.
\begin{equation} \label{diffusion_eq} 
    \frac{v_{gas\, 1}}{v_{gas\, 2}} =
    \sqrt{\frac{M_{gas\, 2}}{M_{gas\, 1}}} =
    \Lambda
\end{equation}
Donde:
\begin{itemize}
    \item $v_{gas\, 1}, v_{gas\, 2}$: Velocidades de difusión de los gases.
    \item $M_{gas\, 1}, M_{gas\, 2}$: Masas molares de los gases.
    \item $\Lambda$: Coeficiente de difusión de Graham.
\end{itemize}

\subsection{Porcentaje de Aproximación}
Se define como:
\[ \% \; aproximaci\acute{o}n = \frac{valor\; real}{valor\; te\acute{o}rico} \cdot 100\%\]
Sirve para cuantificar la aproximación de un valor medido a uno teórico.
\end{document}