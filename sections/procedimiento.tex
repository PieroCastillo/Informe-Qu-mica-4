\documentclass[../main.tex]{subfiles}
\usepackage{float}
\usepackage{tikz}
\usetikzlibrary{positioning, shapes, shadows}
\usepackage{enumitem}
\usepackage{caption}
\captionsetup[table]{name = tabla}

\begin{document}
\begin{itemize}
    \item Experimento 1:
    \begin{figure}[H]
        \tikzstyle{azul}=[rectangle, draw=black, rounded corners, fill=blue!40, drop shadow, text centered, anchor=north, text=black, text width=3cm]
    
        \tikzstyle{rojo}=[rectangle, draw=black, rounded corners, fill=red!40, drop shadow, text centered, anchor=north, text=black, text width=3cm]
    
        \tikzstyle{verde}=[rectangle, draw=black, rounded corners, fill=green!60, drop shadow, text centered, anchor=north, text=black, text width=2cm]
    
        \tikzstyle{amarillo}=[rectangle, draw=black, rounded corners, fill=yellow!60, drop shadow, text centered, anchor=north, text=black, text width=4cm]
        
        \tikzstyle{linea}=[-]
        
        \centering
        \begin{tikzpicture}[align=center, node distance=2cm]
        
        \node (title1) [azul]{Volumen molar del hidrógeno};
    
        \node (clase1) [rojo, below left=1cm and 1.0cm of title1]{Se determinó el volumen muerto de una bureta de 25 ml};
    
        \node (clase2) [rojo, below right=1cm and 1.0cm of title1]{Se colocó cinta de magnesio en un corcho para tapar la bureta};
    
        \draw[linea] (title1.south) -| (clase1.north); 
        \draw[linea] (title1.south) -| (clase2.north); 
    
        \node (clase3) [amarillo, below=4cm of title1]{Se agregó en la bureta 10 ml de HCL 6M + agua destilada};
    
        \node (clase4) [amarillo, below=6cm of title1]{Se introdujo en un vaso de 250ml que contiene agua de caño};
    
        \draw[linea] (clase1.south) |- (clase3.west); 
        \draw[linea] (clase2.south) |- (clase3.east); 
        \draw[linea] (clase3.south) -- (clase4.north); 
        \end{tikzpicture}
        
        \caption{Volumen molar del hidrogeno}
        \label{fig:my_label}
    \end{figure}
    \item Experimento 2:
    \begin{figure}[H]
        \tikzstyle{azul}=[rectangle, draw=black, rounded corners, fill=blue!40, drop shadow, text centered, anchor=north, text=black, text width=3cm]
    
        \tikzstyle{rojo}=[rectangle, draw=black, rounded corners, fill=red!40, drop shadow, text centered, anchor=north, text=black, text width=3cm]
    
        \tikzstyle{verde}=[rectangle, draw=black, rounded corners, fill=green!60, drop shadow, text centered, anchor=north, text=black, text width=2cm]
    
        \tikzstyle{amarillo}=[rectangle, draw=black, rounded corners, fill=yellow!60, drop shadow, text centered, anchor=north, text=black, text width=4cm]
        
        \tikzstyle{linea}=[-]
        
        \centering
        \begin{tikzpicture}[align=center, node distance=2cm]
        
        \node (title1) [azul]{Estudio de la Ley de difusión de Graham};
        
        \node (clase1) [amarillo, below=1cm of title1]{Tubo de vidrio limpio y seco de 25cm aprox};
    
        \node (clase2) [amarillo, below=1cm of clase1]{En los extremos del tubo se colocó simultaneamente dos algodones de NH3(ac) 15 M y de HCL 12 M.};
    
        \node (clase3) [amarillo, below=1cm of clase2]{Se formó un aro blanco de cloruro de amonio};
        
        \draw[linea] (title1.south) -- (clase1.north); 
        \draw[linea] (clase1.south) |- (clase2.north);
        \draw[linea] (clase2.south) |- (clase3.north);
        \end{tikzpicture}
        
        \caption{Estudio de la Ley de difusión de Graham}
        \label{fig:my_label}
    \end{figure}
\end{itemize}
\end{document}